% ****** Note template ****** %

%\RequirePackage{lineno} 

\documentclass[aps,prd,superscriptaddress,floatfix]{revtex4}


\usepackage{graphicx}  % needed for 
\usepackage{dcolumn}   % needed for some tables
\usepackage{bm}        % for math
\usepackage{amssymb}   % for math

%\def\linenumberfont{\normalfont\small\sffamily}

\usepackage{psfrag}


\def\pp{$p\bar{p}$}


\topmargin=-1.1cm

\begin{document}

%\pacs{}


\title{  
\vspace{0.5cm}
Planar Silicon Detector Geometry Description
}

\author {Matthew Herndon}

\address{University of Wisconsin, Madison, Wisconsin, USA}

%\date{\today}

\begin{abstract}
\vskip 0.5cm
\noindent
This document describes the geometry class used to describe a planer silicon strip
detector that measures charged particle trajectories two dimensions immersed in a magnetic field.  
\end{abstract}
\maketitle

%\tableofcontents
%\setpagewiselinenumbers
%%%\modulolinenumbers[10]
%%%\linenumbers

\vspace{0.3cm}

\section{Introduction}
The detector consists of a five layer silicon detector immersed in a 1
Tesla magnetic field oriented perpendicular to the measurement
directions of the silicon sensors.  The interaction vertex is at 0,0,0 though
the program is general enough to support different vertex position.  Particles
generally move in the positive y direction.  The sensors are planes in x,z with
the normal along the y axis, though an arbitrary normal is supported.  Measurements,
are made in 2D as the y position of the sensor and the x position of the strips which
run parallel to the z axis.  The magnetic field is oriented in the z direction to give
bending in the x,y plane.  The detectorGeometry initialization only supports a magnetic 
field along z and sensors normal to the y axis though this could be easily expanded.
The sensors closest to the iteration region are smaller in x and have finer segmentation.

This detector could represent a set of rectangular planes used in a fix target experiment
or a single wedge in phi of the cylindrical collider experiment, where additional identical
wedges could populate the cylinder to complete the cylindrical design.

\section{Units}
 The detector and magnetic field and calculation of trajectories and particle properties are set up such that the units are distance (m), Energy (GeV), magnetic field (Tesla).


\section{Detector layout and performance}
The silicon strip sensors are oriented perpendicular to the y axis in
the x-z plane in 2 cm intervals.  Strips in the sensors run in the z
direction.  The two inner, lowest y,  sensors have strips spaced 25 micron
intervals in x. While the 3 outer sensors have strips spaced in 50 micron
intervals in x.  Each sensor has 2048 strips symmetrically positioned
around x = 0.  The magnetic filed is oriented in the z direction such
that each sensor makes measurements of the x-y position allowing
the curvature in the magnetic field to be measured.

The sensors digitize approximately 32 ADC counts of charge per charged
particle hit and have a hit resolution given in microns which is due to
digitization and intrinsic resolution uncertainties.

The sensors are described by yposition, stripPtich, number of strips
and resolution as give in table~\ref{tab:detectorTable}



\begin{table}
\caption{\label{tab:detectorTable} Sensor properties.}
\begin{tabular}{|l|l|l|l|l|}
\hline 
Layer & Number Strips & Strip Pitch (um) & Y Pos (cm) & Res (um)\\
\hline
0 & 2048 & 25	& 2.0 & 7	 \\
2 & 2048 & 25	& 4.0 & 7	 \\
3 & 2048 & 50	& 6.0 & 12	 \\
4 & 2048 & 50	& 8.0 & 12	 \\
5 & 2048 & 50	& 10.0 & 12	 \\
\hline
\end{tabular}
\end{table}

\section{Class Description}
The detector geometry is available through the detectorGeometry class
described in the detectorGeometry.cc and detectorGeomergy.hh classes.

Class variables:
\begin{itemize}
\item int nSensors: number for sensors in y
\item double bField: magnetic field strength, oriented along z-axis
\item sensorDecriptor structs: describing silicon sensors
\end{itemize}




sensorDecriptor struct:\\
Instances sensor0, sensor1, sensor2, sensor3, sensor4.\\
Variables:\\
\begin{itemize}
\item int nStrips
\item double stripPitch
\item double yPos
\item double resolution
\end{itemize}



% \begin{thebibliography}{99}

% \bibitem{higgs}
% P.W. Higgs, Phys. Rev. Lett. 12, 132 (1964),
% {\it idem\rm}, Phys. Rev. 145, 1156 (1966),
% F. Englert and R. Brout, Phys. Rev. Lett. 13, 321 (1964), 
% G.S. Guralnik, C.R. Hagen, and T.W. Kibble, Phys. Rev. Lett. 13, 585 (1964)


% \end{thebibliography}
% 
\end{document}