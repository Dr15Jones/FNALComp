% ****** Note template ****** %

%\RequirePackage{lineno} 

\documentclass[aps,prd,superscriptaddress,floatfix]{revtex4}


\usepackage{graphicx}  % needed for 
\usepackage{dcolumn}   % needed for some tables
\usepackage{bm}        % for math
\usepackage{amssymb}   % for math

%\def\linenumberfont{\normalfont\small\sffamily}

\usepackage{psfrag}


\def\pp{$p\bar{p}$}


\topmargin=-1.1cm

\begin{document}

%\pacs{}


\title{  
\vspace{0.5cm}
Description of the Fermilab software school data objects
}

\author {Matthew Herndon}

\address{University of Wisconsin, Madison, Wisconsin, Fermi
  National Accelerator Laboratory, Illinois}

%\date{\today}

\begin{abstract}
\vskip 0.5cm
\noindent
This document describes the data objects use in the Fermilab software school.
The objects include a StripSet with raw digitized data for the strip sensors
of the top simulated detector and Hits, Tracks (and Helix), GenTracks and associated
HitSets, TracksSets and GenTrackSets.  In addition the format used to store the
data objects to file is described.
\end{abstract}
\maketitle

%\tableofcontents
%\setpagewiselinenumbers
%%%\modulolinenumbers[10]
%%%\linenumbers

\vspace{0.3cm}

\section{Introduction}
Each objects is described below including the objects class and the methods and
format used to write the objects.  Sets are described with the associated data
object.

The detector geometry of the strip sensors is defined in ~\cite{detectorGeometry}.


\section{The StripSet}
\subsection(StripSet class)
Implemented in the StripSet.hh and StripSet.cc files in the DataObjects directories.

The stripSet is a std::vector of typedef layerStripMaps with one map per layers of the detector.
The detector is dynamically defined from a geometry files and the vector is initialized
at run time.  Each layerStripMap is a map of key<int> stripNumber and adc value.  This allows
the strip data to be stored in a sparse and ordered way.

The strip data can be accessed using the generic getStrips() method which gives you the entire
vector. The underlying vector container could be changed several different container types without
effecting the user interface.  The strip data can also be accessed using the getLayerStripMap(int layer)
method for direct access to layer map.  The method indicates that a Map is retrieved since Maps have
access methods methods that are not common to containers.

An insertStrip(unsigned int layer, int stripNumber, int acd) method is provided to insert data into
the set.

The StripSet includes print(ostream&) to print to an a user specified ostream.

\subsection{StripSet IO)
Implemented in the StripSetIO.hh and StripSetIO.cc files in the Algorithms directories.

Input and output to disk in controlled by the StripSetIO helper class.  Reading and writing
are performed by the readEvent and writeEvent functions.

The data is stored, or ``streamed'' to disk in bit packed format to illustrate a typical way date is stored in compact
format in many experiments.  The =data is stored in individual bytes or pairs of bytes for information
that does not fit in one byte.  The data structure is as follows.

\begin{itemize}
\item Strips: in clean test to identify the Set
\item  1 byte: version The version is checked on input to make sure the correct streamer code is being used.
\item - 1 byte: layer number, current 0-9 (repeated once per layer )
\item -- 1 byte: number of strips, max 256
\item --- 2 bytes: strip number, 11 bytes max 2048, and acd 5 bytes, max 32
  (repeated number of strips times)
\end{itemize}

Note event numbers and layer numbers are not necessary to the
structure but included to facilitate synchronization with write and read operation..

A number of helper functions exist in the StripHitFunctions file in Geometry directories.  This functions
can convert from strip number and layer to local and global positions and identify whether a strip number is valid
on a given layer.

\subsection{binary input and output}
Data input and output is controlled via standard C++ library binary input and
output functions. std::ofstream, ifstream.  Byte writing is performed
write and read function as:

stripdata.write (reinterpret\_cast<const char *>(\&myInt), 1);\\
stripdata.read (myCharByte, 1);\\

\section{The Hit and HitSet}
\subsection{Hit class)
Implemented in the Hit.hh and Hit.cc files in the DataObjects directories.

The Hit class contains:\\
TVector3 _hitPosition\\
int _layer
int _numberStrips\\
Int _trackNumber (only for generator level hits)\\

The class servers for both generator Hits and reconstructed Hits.   Associated information such as Hit resolutions
are stored in the DetectorGeometry.  Also note that a hit can be though of as ``bad'' if it has a large ADC
value or more than two associated strips indicating that it is actually due to several overlapping hits from
different tracks.

Each characteristic of the Hit can be accessed by a got function of the same name: getHitPosition().

Hits are constructed by hitPosition,layer,numberStrips,trackNumber.

includes print(ostream&) to print to an a user specified ostream.



\subsection(HitSet class)
Implemented in the HitSet.hh and HitSet.cc files in the DataObjects directories.

The HitSet is a typedef hitSet of type std::vector of Hits.

The hit data can be accessed using the generic getHits() method which gives you the entire
vector. The vector index indexes the Hit number to allow direct access to a hit of a known number.


An insertHit(Hit) method is provided to insert data into the set.


The HitSet includes print(ostream&) to print to an a user specified ostream that iterates over
the Hits calling the print method of each Hit..

A number of helper functions exist in the StripHitFunctions file in Geometry directories.  These functions
can convert from hit position to local and strip number position  identify whether a hit position at a sensor
plane is within the active area of the sensor.



\subsection{HitSet IO)
Implemented in the HitSetIO.hh and HitSetIO.cc files in the Algorithms directories.

Input and output to disk in controlled by the HitSetIO helper class.  Reading and writing
are performed by the readEvent and writeEvent functions.

The data is stored in text format for simplicity.


\begin{itemize}
\item  Hits: to identify the set
\item version: The version is checked on input to make sure the correct streamer code is being used.
\item number of hits
\item hit number (repeated number of hits times)
\item x position
\item y position
\item z position
\item layer number
\item number of strips
\item track number, -1 if none 
\end{itemize}

\section{The GenTrack and GenTrackSet}
\subsection{GenTrack class)
Implemented in the GenTrack.hh and GenTrack.cc files in the DataObjects directories.

The GenTrack class contains:\\
TLorentzVector _lorentzVector\\
int _charge
int _dr, point of closest approach to the reference point, 0,0,0\\

Each characteristic of the Hit can be accessed by a got function of the same name: getPosition() for _dr.
There is also a makeHelix(alpha) function which takes the curvature constant in the magnetic field and returns
a Helix which can be used for operations like finding intersections with layers.  Note the magnetic field
value is needed since it may not be uniform and the helix parameters would be different at different points.

Gentracks are constructed by TLorentzVector,int charge, and TVector3 position of closest approach.

Gentrack includes print(ostream&) to print to an a user specified ostream.


\subsection(GenTrackSet class)
Implemented in the GenTrackSet.hh and GenTrackSet.cc files in the DataObjects directories.

The GenTrackSet is a typedef genTrackSet of type std::vector of GenTracks.

The GenTrack data can be accessed using the generic getGenTracks() method which gives you the entire
vector. The vector index indexes the GenTrackNumber number to allow direct access to a GenTrack of a known number.


An insertTrack(GenTrack) method is provided to insert data into the set.


The GenTrackSet includes print(ostream&) to print to an a user specified ostream that iterates over
the GenTracks calling the print method of each GenTrack..

\subsection{GenTrackSet IO)
Implemented in the GenTrackSetIO.hh and GenTrackSetIO.cc files in the Algorithms directories.

Input and output to disk in controlled by the GenTrackSetIO helper class.  Reading and writing
are performed by the readEvent and writeEvent functions.

The data is stored in text format for simplicity.


\begin{itemize}
\item  GenTracks: to identify the set
\item version: The version is checked on input to make sure the correct streamer code is being used.
\item number of GenTracks
\item GenTrack number (repeated number of GenTrackNumber times)
\item charge
\item px
\item py
\item pz
\item E
\item x position
\item y position
\item z position
\end{itemize}

\section{The GenTrack and GenTrackSet}
\subsection{Track class)
Implemented in the Track.hh and Track.cc files in the DataObjects directories.


The Track class contains:\\
Helix _helix, 5 track parameter helix\\
int _covMatrix, 5x5 parameter covarinace matrix\\
double _chi2\\
int _nDof\\
_trackHitSet a typedef vector type of ints that are hit index numbers.

The Track and Helix objects are described in more detail in the Track.pdf document.

Tracks are constructed by the full set of parameters above.  These parameters
are generated by the BuildTrack helper function in the Algorithms directories.
That class fits a track based on a number of hits.

Track includes print(ostream&) to print to an a user specified ostream.


\subsection(TrackSet class)
Implemented in the TrackSet.hh and TrackSet.cc files in the DataObjects directories.

The TrackSet is a typedef trackSet of type std::vector of GenTracks.

The Track data can be accessed using the generic getTracks() method which gives you the entire
vector. The vector index indexes the TrackNumber number to allow direct access to a Track of a known number.

An insertTrack(Track) method is provided to insert data into the set.


The TrackSet includes print(ostream&) to print to an a user specified ostream that iterates over
the Tracks calling the print method of each Track..

\subsection{TrackSet IO)
Implemented in the TrackSetIO.hh and TrackSetIO.cc files in the Algorithms directories.

Input and output to disk in controlled by the TrackSetIO helper class.  Reading and writing
are performed by the readEvent and writeEvent functions.

The data is stored in text format for simplicity.


\begin{itemize}
\item  GenTracks: to identify the set
\item version: The version is checked on input to make sure the correct streamer code is being used.
\item number of GenTracks
\item GenTrack number (repeated number of GenTrackNumber times)
\item charge
\item px
\item py
\item pz
\item E
\item x position
\item y position
\item z position
\item number of Hits
\item Hit index numbers (repeated number of Hits times)
\end{itemize}

The hit list could be used alone to restore the track.  Currently TrackSetIO is not used.


\begin{thebibliography}{99}

\bibitem{detectorGeometry}
detectorGeometry.pdf note, M. Herndon (2014)

\bibitem{detectorGeometry}
track.pdf note, M. Herndon (2014)

\end{thebibliography}
% 
\end{document}