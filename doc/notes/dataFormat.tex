% ****** Note template ****** %

%\RequirePackage{lineno} 

\documentclass[aps,prd,superscriptaddress,floatfix]{revtex4}


\usepackage{graphicx}  % needed for 
\usepackage{dcolumn}   % needed for some tables
\usepackage{bm}        % for math
\usepackage{amssymb}   % for math

%\def\linenumberfont{\normalfont\small\sffamily}

\usepackage{psfrag}


\def\pp{$p\bar{p}$}


\topmargin=-1.1cm

\begin{document}

%\pacs{}


\title{  
\vspace{0.5cm}
Planar Silicon Detector Data Format
}

\author {Matthew Herndon}

\address{University of Wisconsin, Madison, Wisconsin, Fermi
  National Accelerator Laboratory, Illinois}

%\date{\today}

\begin{abstract}
\vskip 0.5cm
\noindent
This document describes the data format and classes used
to store data for a planer silicon strip detector.
\end{abstract}
\maketitle

%\tableofcontents
%\setpagewiselinenumbers
%%%\modulolinenumbers[10]
%%%\linenumbers

\vspace{0.3cm}

\section{Introduction}
Digitized data from the planer silicon strip detector is stored
in bit packed format.

The detector geometry of the silicon sensors is defined in ~\cite{detectorGeometry}.


\section{Data Format}
Data is stored in individual bytes or pairs of bytes for information
that does not fit in one byte.  The data structure is as follows.

\begin{itemize}
\item  1 byte: event number, max 256
\item - 1 byte: layer number, max 5 (repeated number of events times)
\item -- 1 byte: number of strips, max 256 (repeated 5 times)
\item --- 2 bytes: strip number, 11 bytes max 2048, 5 bytes, max 32
  (repeated number of strips times)
\end{itemize}

Note event numbers and layer numbers are not necessary to the
structure but included to facilitate synchronization with reading of
other event information and with layer ordering.

\section{Data input and output}
Data input and output is controlled via standard C++ library binary input and
output functions. std::ofstream, ifstream.  Byte writing is performed
write and read function as:

stripdata.write (reinterpret\_cast<const char *>(\&myInt), 1);\\
stripdata.read (myCharByte, 1);\\

A planned upgrade is to hide the read and write implementations to allow the
underlying method to be optimized to improve performance for all
software using the byte input output format.


\section{stripSet class}
Strip data can be stored in the stripSet class implemented in the
stripSet.cc and stripSet.hh files.

Class variables:
\begin{itemize}
\item type\_def map layer\_map: int key int value map of strip and adc data, with
\item vector layer\_map\_vector: vector of number of sensor layers layer\_maps
\end{itemize}

Class functions:
\begin{itemize}
\item void clear(void) to clear the layer\_map\_vector
\end{itemize}





\begin{thebibliography}{99}

\bibitem{detectorGeometry}
detectorGeometry note, M. Herndon (2014)

\end{thebibliography}
% 
\end{document}